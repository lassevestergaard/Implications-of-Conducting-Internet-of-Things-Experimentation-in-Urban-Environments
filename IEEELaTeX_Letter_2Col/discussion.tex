\section{discussion}
Through our tracking experiments we encountered several challenges which affected the quality of the technical setup as well as the following data analysis. From the physical setup perspective, we had to make compromises which were assisted by the fact that the festival granted a unique access to festival grounds and installations including power grid.

For both festivals two challenges were encountered that made it imperative to make some informed choices: Firstly, it was decided to use 4G dongles to avoid any issues with the festival’s own wired infrastructure. This decision was twofold: 1) If the festival experienced any issues with their connectivity it could inflict badly on the data gathering. 2) The sensors could run the risk of introducing potential problems in the Point-Of-Service setup that could have devastating effects on the festival's economy. Secondly, the physical layout of the festivals gave rise to some challenges as both festivals have large vacant areas in the center of the grounds to afford audiences a better view of stages and move about, which hamstrung the effort to position the sensors in the optimal locations. This resulted in a loss of granularity and coverage, but could not be avoided unless dedicating significant resources to setting up e.g. battery stations. We therefore focused on the accompanying elements, such as bars, food stands, access areas and toilets as our major focal point.

During Northside Festival, we decided against collecting spoofed data points as it was assumed that this could jeopardize the system’s ability to handle the flow, which was expected to include around 40.000 devices inside the range of normal opening hours. Our hypothesis was validated, when activating spoofed data for a short time span during peak hours at the festival, which resulted in a break down (see label '08/06 15:00' in Figure~\ref{fig:stationary_noise}). Collecting and transmitting all this data could pose a threat to stability of the servers as well as create buffering issues on the devices as the 4G dongles had a limit, especially with many people in the vicinity, using their mobile phones. During Haven Festival, a much lower number was expected and was deemed possible for the system to handle this pressure, and therefore the collection of spoofed Wi-Fi data points was activated to facilitate further post-analysis on these. 

From the data analysis perspective we learned that finding a good trade-off between coverage and granularity, and having both at a high enough level is essential to be able to analyze how visitors move around at the festival area. Nevertheless, it is possible to remove uninformative or spurious data when using sub-optimal setups. Furthermore, being able to group the devices into guests and employees (see Figure~\ref{fig:features} D) can be helpful to perform a different analysis per group. This, however, would require that sensors are set up in areas where only employees or volunteers have access to distinguish those devices from the rest.

The overarching learnings suggest that the physical layout of the festivals seems to be the culprit for our poor results. This further indicates that relatively mundane aspects like access to physical infrastructures can have significant impact on the ability to conduct IoT experiments in urban environments. Going back to the idea of hackable cities, this is an interesting movement with the ability to revolutionize real- and large-scale IoT prototyping through an \textit{always on} platform accessible for anyone independently of affiliation and available economic resources. In order to unlock the real potential, we argue that cities are not hackable by default, but they need to be carefully designed and reshaped in order to enable high quality experimentation which again feeds into collaborative city-making.