\section{Related Work}

Through our investigations, we have come across research on urban experimentation, where especially~\cite{2019} is central. The authors emphasize city hacking as a platform for tinkering and experimenting for the greater good of cities and citizens at large. The pivot point is mainly addressing hackability as a tool for engagement and collaboration, and not the actual technical implications of conducting experiments. Others address learnings from conducting actual urban experiments and practical implications~\cite{Amaxilatis2018}. Their focus is primarily related to a specific experimentation platform developed during a three year EU project, and does not address the implications of utilizing the built environment. Other research is focusing on aspects like Maker cultures, where emphasis is put on the design approach surrounding the developed artefacts rather than the practical testing and resulting implications~\cite{9780071821131}. Nevertheless, practical technical implications have been addressed and e.g.~\cite{Longo:2017:CDC:3155100.3093895} have elaborated the need for calibrating sensors and managing the issues of poor measurement quality.

From the technical perspective, we have encountered examples of tracking people's movement. These utilize a number of different technologies to uniquely identify a person. Some novel approaches use behavioral features, such as gait~\cite{gait} or face recognition~\cite{face}. Others rely on cybernetic interfaces such as mobile phones or beacons to track people. Included in this term we see examples of using GPS chips in the phone~\cite{gps}, Bluetooth signal~\cite{bluetooth} or Wi-Fi probe requests~\cite{wifihubei,wifiuti,wifimonitor}.
In our research, we have utilized the latter to extrapolate information from the collected data. Finally, we are leveraging the methodology from~\cite{semantic} to make segmentations based on features extracted from spatial and temporal information about devices~\cite{largescalemonitoring, monitorflows}.