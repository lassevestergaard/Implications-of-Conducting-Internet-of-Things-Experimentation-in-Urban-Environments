\section{Introduction}

Tinkering with Information Technology (IT) and developing prototypes is happening everywhere, and is becoming a basic skill, as well as an approach when diving into uncharted technical territories. In parallel to this, cities are subject to similar movements, and are now being portrayed as ever changing domains that are never complete - they are perpetual betas. Consequently, cities are turning into dynamic environments that are fostering and even encouraging hacking and prototyping~\cite{Fredericks2019}.

City authorities have started to exploit the possibilities unfolding from cities being temporary, adjustable and hackable and they are promoting cities as test beds and environments for experimentation~\cite{Latre2016}. Cities are constantly being augmented with new IT infrastructures (e.g. Wi-Fi, Low Power Wide Area Networks (LPWAN) and intelligent lamp posts), advancing possibilities for experimentation -- we now talk about cities as platforms for hacking. In addition, infrastructures are being prepared for deploying all kinds of temporal equipment like sensors, screens, interactive installations etc. By adopting the notion of city as a platform, the urban environment becomes a common ground between hackers and authorities, where they can synergize on city development through a combination of top-down and bottom-up approaches~\cite{Mulder2019}. In our research, we are following the trend of hackable cities~\cite{2019}, and are investigating how to utilize such platforms for conducting urban experiments. We are approaching this arena from a technically oriented Internet of Things (IoT) \cite{Ghasempour2016} prototyping perspective, and are especially interested in the practical implications that emerge when hacking the urban environment.

Even though we are tapping into the hackable city movement, we have chosen not to focus on actual cities as our physical testing ground. Instead we have perceived festivals as temporal cities. A festival has the same basic infrastructure elements like electricity grid, water supply, draining systems and waste management. It often has residential areas (camping areas), shopping areas and leisure zones (including concert areas). Last but not least, festivals represent physical areas with a high concentration of moving and interacting people~\cite{Jarvis:2013:USW:2494091.2499216}. City centres usually experience a constant flow of people traversing the streets, but the concentration of people is governed by factors such as day of week, time of day, city events and season. Consequently, it can be challenging to find the optimal occasion for conducting IoT experiments like tracking people. Furthermore, obtaining permission to conduct an IoT experiment in a city centre, can be time consuming due to bureaucratic city administrations. While it depends on the actual aim of a specific experiment, we wanted to test Wi-Fi tracking abilities in crowded environments. Additionally, we needed a critical amount of tracking data in order to make reliable post-analysis. Adding up these prerequisites, the focus was moved towards festivals instead of conventional cities, as we concluded that the likelihood of experiencing an optimal environment was higher in a physically confined environment.

From an overarching perspective, this article is investigating the potential of IoT prototyping in urban environments. We want to find out if state-of-the-art IoT technologies (specifically Wi-Fi dongles and Raspberry Pis) can be reliably utilized under prototyping conditions in urban settings in order to understand practical implications of temporarily deploying IoT prototypes in such settings.

In the remainder of the article we initially introduce related work. Then we elaborate the research methodology, and describe the technical setup, of the prototyping environment, including data analysis considerations. This is followed up by a case study from two festivals in Denmark, and finally we discuss practical learnings.