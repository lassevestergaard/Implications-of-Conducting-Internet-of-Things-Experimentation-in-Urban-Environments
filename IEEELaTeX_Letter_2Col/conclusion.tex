\section{Conclusion}
In this article we have been focusing on practical learnings from conducting IoT prototyping in hackable urban environments. We have specifically emphasized festival settings, the technical setup, data analysis and general learnings. The state-of-the-art IoT technology we have utilized in our research has performed as expected, and did not, in itself, result in major breakdowns during the deployment phase. We did learn that the physical infrastructure of the festivals were challenging because Wi-Fi tracking was not taken into account when designing the festival layout. Additionally, the settings were optimized for best audience experiences, and not hacking in general. These aspects are evident in our post data analysis, and made it challenging to conduct high quality analysis due to holes in data as well as difficulties tracking devices across the entire festival area. We experienced noisy data due to sub-optimal locations of Wi-Fi trackers (e.g. near exits and thereby collecting data from bypassers outside of the festival premise). 
Even though the quality of the hardware and software were sufficient, it is important to note that the overall quality of our data could have been better. This was mainly due to a discrepancy between the hypotheses from the festivals and the feasibility of the physical environment to cover these. This goes to show that we need to design future hackable cities carefully, and on a practical level, be able to support placement of equipment in a multitude of ways as to support different types of experiments as the physical location can have great impact on the results.